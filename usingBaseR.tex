% Options for packages loaded elsewhere
\PassOptionsToPackage{unicode}{hyperref}
\PassOptionsToPackage{hyphens}{url}
\PassOptionsToPackage{dvipsnames,svgnames,x11names}{xcolor}
%
\documentclass[
  letterpaper,
  DIV=11,
  numbers=noendperiod]{scrartcl}

\usepackage{amsmath,amssymb}
\usepackage{iftex}
\ifPDFTeX
  \usepackage[T1]{fontenc}
  \usepackage[utf8]{inputenc}
  \usepackage{textcomp} % provide euro and other symbols
\else % if luatex or xetex
  \usepackage{unicode-math}
  \defaultfontfeatures{Scale=MatchLowercase}
  \defaultfontfeatures[\rmfamily]{Ligatures=TeX,Scale=1}
\fi
\usepackage{lmodern}
\ifPDFTeX\else  
    % xetex/luatex font selection
\fi
% Use upquote if available, for straight quotes in verbatim environments
\IfFileExists{upquote.sty}{\usepackage{upquote}}{}
\IfFileExists{microtype.sty}{% use microtype if available
  \usepackage[]{microtype}
  \UseMicrotypeSet[protrusion]{basicmath} % disable protrusion for tt fonts
}{}
\makeatletter
\@ifundefined{KOMAClassName}{% if non-KOMA class
  \IfFileExists{parskip.sty}{%
    \usepackage{parskip}
  }{% else
    \setlength{\parindent}{0pt}
    \setlength{\parskip}{6pt plus 2pt minus 1pt}}
}{% if KOMA class
  \KOMAoptions{parskip=half}}
\makeatother
\usepackage{xcolor}
\setlength{\emergencystretch}{3em} % prevent overfull lines
\setcounter{secnumdepth}{-\maxdimen} % remove section numbering
% Make \paragraph and \subparagraph free-standing
\makeatletter
\ifx\paragraph\undefined\else
  \let\oldparagraph\paragraph
  \renewcommand{\paragraph}{
    \@ifstar
      \xxxParagraphStar
      \xxxParagraphNoStar
  }
  \newcommand{\xxxParagraphStar}[1]{\oldparagraph*{#1}\mbox{}}
  \newcommand{\xxxParagraphNoStar}[1]{\oldparagraph{#1}\mbox{}}
\fi
\ifx\subparagraph\undefined\else
  \let\oldsubparagraph\subparagraph
  \renewcommand{\subparagraph}{
    \@ifstar
      \xxxSubParagraphStar
      \xxxSubParagraphNoStar
  }
  \newcommand{\xxxSubParagraphStar}[1]{\oldsubparagraph*{#1}\mbox{}}
  \newcommand{\xxxSubParagraphNoStar}[1]{\oldsubparagraph{#1}\mbox{}}
\fi
\makeatother

\usepackage{color}
\usepackage{fancyvrb}
\newcommand{\VerbBar}{|}
\newcommand{\VERB}{\Verb[commandchars=\\\{\}]}
\DefineVerbatimEnvironment{Highlighting}{Verbatim}{commandchars=\\\{\}}
% Add ',fontsize=\small' for more characters per line
\usepackage{framed}
\definecolor{shadecolor}{RGB}{241,243,245}
\newenvironment{Shaded}{\begin{snugshade}}{\end{snugshade}}
\newcommand{\AlertTok}[1]{\textcolor[rgb]{0.68,0.00,0.00}{#1}}
\newcommand{\AnnotationTok}[1]{\textcolor[rgb]{0.37,0.37,0.37}{#1}}
\newcommand{\AttributeTok}[1]{\textcolor[rgb]{0.40,0.45,0.13}{#1}}
\newcommand{\BaseNTok}[1]{\textcolor[rgb]{0.68,0.00,0.00}{#1}}
\newcommand{\BuiltInTok}[1]{\textcolor[rgb]{0.00,0.23,0.31}{#1}}
\newcommand{\CharTok}[1]{\textcolor[rgb]{0.13,0.47,0.30}{#1}}
\newcommand{\CommentTok}[1]{\textcolor[rgb]{0.37,0.37,0.37}{#1}}
\newcommand{\CommentVarTok}[1]{\textcolor[rgb]{0.37,0.37,0.37}{\textit{#1}}}
\newcommand{\ConstantTok}[1]{\textcolor[rgb]{0.56,0.35,0.01}{#1}}
\newcommand{\ControlFlowTok}[1]{\textcolor[rgb]{0.00,0.23,0.31}{\textbf{#1}}}
\newcommand{\DataTypeTok}[1]{\textcolor[rgb]{0.68,0.00,0.00}{#1}}
\newcommand{\DecValTok}[1]{\textcolor[rgb]{0.68,0.00,0.00}{#1}}
\newcommand{\DocumentationTok}[1]{\textcolor[rgb]{0.37,0.37,0.37}{\textit{#1}}}
\newcommand{\ErrorTok}[1]{\textcolor[rgb]{0.68,0.00,0.00}{#1}}
\newcommand{\ExtensionTok}[1]{\textcolor[rgb]{0.00,0.23,0.31}{#1}}
\newcommand{\FloatTok}[1]{\textcolor[rgb]{0.68,0.00,0.00}{#1}}
\newcommand{\FunctionTok}[1]{\textcolor[rgb]{0.28,0.35,0.67}{#1}}
\newcommand{\ImportTok}[1]{\textcolor[rgb]{0.00,0.46,0.62}{#1}}
\newcommand{\InformationTok}[1]{\textcolor[rgb]{0.37,0.37,0.37}{#1}}
\newcommand{\KeywordTok}[1]{\textcolor[rgb]{0.00,0.23,0.31}{\textbf{#1}}}
\newcommand{\NormalTok}[1]{\textcolor[rgb]{0.00,0.23,0.31}{#1}}
\newcommand{\OperatorTok}[1]{\textcolor[rgb]{0.37,0.37,0.37}{#1}}
\newcommand{\OtherTok}[1]{\textcolor[rgb]{0.00,0.23,0.31}{#1}}
\newcommand{\PreprocessorTok}[1]{\textcolor[rgb]{0.68,0.00,0.00}{#1}}
\newcommand{\RegionMarkerTok}[1]{\textcolor[rgb]{0.00,0.23,0.31}{#1}}
\newcommand{\SpecialCharTok}[1]{\textcolor[rgb]{0.37,0.37,0.37}{#1}}
\newcommand{\SpecialStringTok}[1]{\textcolor[rgb]{0.13,0.47,0.30}{#1}}
\newcommand{\StringTok}[1]{\textcolor[rgb]{0.13,0.47,0.30}{#1}}
\newcommand{\VariableTok}[1]{\textcolor[rgb]{0.07,0.07,0.07}{#1}}
\newcommand{\VerbatimStringTok}[1]{\textcolor[rgb]{0.13,0.47,0.30}{#1}}
\newcommand{\WarningTok}[1]{\textcolor[rgb]{0.37,0.37,0.37}{\textit{#1}}}

\providecommand{\tightlist}{%
  \setlength{\itemsep}{0pt}\setlength{\parskip}{0pt}}\usepackage{longtable,booktabs,array}
\usepackage{calc} % for calculating minipage widths
% Correct order of tables after \paragraph or \subparagraph
\usepackage{etoolbox}
\makeatletter
\patchcmd\longtable{\par}{\if@noskipsec\mbox{}\fi\par}{}{}
\makeatother
% Allow footnotes in longtable head/foot
\IfFileExists{footnotehyper.sty}{\usepackage{footnotehyper}}{\usepackage{footnote}}
\makesavenoteenv{longtable}
\usepackage{graphicx}
\makeatletter
\newsavebox\pandoc@box
\newcommand*\pandocbounded[1]{% scales image to fit in text height/width
  \sbox\pandoc@box{#1}%
  \Gscale@div\@tempa{\textheight}{\dimexpr\ht\pandoc@box+\dp\pandoc@box\relax}%
  \Gscale@div\@tempb{\linewidth}{\wd\pandoc@box}%
  \ifdim\@tempb\p@<\@tempa\p@\let\@tempa\@tempb\fi% select the smaller of both
  \ifdim\@tempa\p@<\p@\scalebox{\@tempa}{\usebox\pandoc@box}%
  \else\usebox{\pandoc@box}%
  \fi%
}
% Set default figure placement to htbp
\def\fps@figure{htbp}
\makeatother

\KOMAoption{captions}{tableheading}
\makeatletter
\@ifpackageloaded{caption}{}{\usepackage{caption}}
\AtBeginDocument{%
\ifdefined\contentsname
  \renewcommand*\contentsname{Table of contents}
\else
  \newcommand\contentsname{Table of contents}
\fi
\ifdefined\listfigurename
  \renewcommand*\listfigurename{List of Figures}
\else
  \newcommand\listfigurename{List of Figures}
\fi
\ifdefined\listtablename
  \renewcommand*\listtablename{List of Tables}
\else
  \newcommand\listtablename{List of Tables}
\fi
\ifdefined\figurename
  \renewcommand*\figurename{Figure}
\else
  \newcommand\figurename{Figure}
\fi
\ifdefined\tablename
  \renewcommand*\tablename{Table}
\else
  \newcommand\tablename{Table}
\fi
}
\@ifpackageloaded{float}{}{\usepackage{float}}
\floatstyle{ruled}
\@ifundefined{c@chapter}{\newfloat{codelisting}{h}{lop}}{\newfloat{codelisting}{h}{lop}[chapter]}
\floatname{codelisting}{Listing}
\newcommand*\listoflistings{\listof{codelisting}{List of Listings}}
\makeatother
\makeatletter
\makeatother
\makeatletter
\@ifpackageloaded{caption}{}{\usepackage{caption}}
\@ifpackageloaded{subcaption}{}{\usepackage{subcaption}}
\makeatother

\usepackage{bookmark}

\IfFileExists{xurl.sty}{\usepackage{xurl}}{} % add URL line breaks if available
\urlstyle{same} % disable monospaced font for URLs
\hypersetup{
  pdftitle={Using BaseR},
  pdfauthor={Mike Keating},
  colorlinks=true,
  linkcolor={blue},
  filecolor={Maroon},
  citecolor={Blue},
  urlcolor={Blue},
  pdfcreator={LaTeX via pandoc}}


\title{Using BaseR}
\author{Mike Keating}
\date{}

\begin{document}
\maketitle


\subsection{Task 1: Basic Vector
Practice}\label{task-1-basic-vector-practice}

\subsubsection{Question 1:}\label{question-1}

Create two vectors named pre and post. One vector corresponding to the
pre measurements and one to the post measurements.. Create two vectors
named pre and post. One vector corresponding to the pre measurements and
one to the post measurements.

\begin{Shaded}
\begin{Highlighting}[]
\NormalTok{pre }\OtherTok{\textless{}{-}} \FunctionTok{c}\NormalTok{(}\DecValTok{130}\NormalTok{, }\DecValTok{128}\NormalTok{, }\DecValTok{116}\NormalTok{, }\DecValTok{124}\NormalTok{, }\DecValTok{133}\NormalTok{, }\DecValTok{134}\NormalTok{, }\DecValTok{118}\NormalTok{, }\DecValTok{126}\NormalTok{, }\DecValTok{114}\NormalTok{, }\DecValTok{127}\NormalTok{,}
         \DecValTok{141}\NormalTok{, }\DecValTok{138}\NormalTok{, }\DecValTok{128}\NormalTok{, }\DecValTok{140}\NormalTok{, }\DecValTok{137}\NormalTok{, }\DecValTok{131}\NormalTok{, }\DecValTok{120}\NormalTok{, }\DecValTok{128}\NormalTok{, }\DecValTok{139}\NormalTok{, }\DecValTok{135}\NormalTok{)}

\NormalTok{post }\OtherTok{\textless{}{-}} \FunctionTok{c}\NormalTok{(}\DecValTok{114}\NormalTok{, }\DecValTok{98}\NormalTok{, }\DecValTok{113}\NormalTok{, }\DecValTok{99}\NormalTok{, }\DecValTok{107}\NormalTok{, }\DecValTok{116}\NormalTok{, }\DecValTok{113}\NormalTok{, }\DecValTok{111}\NormalTok{, }\DecValTok{119}\NormalTok{, }\DecValTok{117}\NormalTok{,}
          \DecValTok{101}\NormalTok{, }\DecValTok{119}\NormalTok{, }\DecValTok{130}\NormalTok{, }\DecValTok{122}\NormalTok{, }\DecValTok{106}\NormalTok{, }\DecValTok{106}\NormalTok{, }\DecValTok{124}\NormalTok{, }\DecValTok{102}\NormalTok{, }\DecValTok{117}\NormalTok{, }\DecValTok{113}\NormalTok{)}
\end{Highlighting}
\end{Shaded}

\subsubsection{Question 2:}\label{question-2}

Assign names to the vector elements using the paste() function. Note
that names() can be overwritten by a character vector.

\begin{Shaded}
\begin{Highlighting}[]
\NormalTok{names }\OtherTok{\textless{}{-}} \FunctionTok{paste}\NormalTok{(}\StringTok{"Subject"}\NormalTok{, }\DecValTok{1}\SpecialCharTok{:}\DecValTok{20}\NormalTok{, }\AttributeTok{sep=}\StringTok{"\_"}\NormalTok{)}
\CommentTok{\# Assign names}
\FunctionTok{names}\NormalTok{(pre) }\OtherTok{\textless{}{-}}\NormalTok{ names}
\FunctionTok{names}\NormalTok{(post) }\OtherTok{\textless{}{-}}\NormalTok{ names}
\end{Highlighting}
\end{Shaded}

\subsubsection{Question 3:}\label{question-3}

Calculate the change in blood pressure for each patient.

\begin{Shaded}
\begin{Highlighting}[]
\CommentTok{\# Change in blood pressure}
\NormalTok{diff\_bp }\OtherTok{\textless{}{-}}\NormalTok{ pre }\SpecialCharTok{{-}}\NormalTok{ post}
\NormalTok{diff\_bp}
\end{Highlighting}
\end{Shaded}

\begin{verbatim}
 Subject_1  Subject_2  Subject_3  Subject_4  Subject_5  Subject_6  Subject_7 
        16         30          3         25         26         18          5 
 Subject_8  Subject_9 Subject_10 Subject_11 Subject_12 Subject_13 Subject_14 
        15         -5         10         40         19         -2         18 
Subject_15 Subject_16 Subject_17 Subject_18 Subject_19 Subject_20 
        31         25         -4         26         22         22 
\end{verbatim}

\subsubsection{Question 4:}\label{question-4}

Calculate the average decrease in blood pressure across all patients.

\begin{Shaded}
\begin{Highlighting}[]
\NormalTok{avg\_change\_all }\OtherTok{\textless{}{-}} \FunctionTok{mean}\NormalTok{(diff\_bp)}
\end{Highlighting}
\end{Shaded}

\subsubsection{Question 5}\label{question-5}

Determine which patients experienced a decrease in blood pressure after
treatment (a positive change). Use the which() function to just return
the indices (and names) associated with this type of change.

\begin{Shaded}
\begin{Highlighting}[]
\NormalTok{index\_patients\_decreased\_bp }\OtherTok{\textless{}{-}} \FunctionTok{which}\NormalTok{(diff\_bp }\SpecialCharTok{\textgreater{}} \DecValTok{0}\NormalTok{)}
\end{Highlighting}
\end{Shaded}

\subsubsection{Question 6}\label{question-6}

Subset the vector of differences to only return those that have a
positive change

\begin{Shaded}
\begin{Highlighting}[]
\NormalTok{patients\_decreased\_bp }\OtherTok{\textless{}{-}}\NormalTok{ diff\_bp[index\_patients\_decreased\_bp]}
\end{Highlighting}
\end{Shaded}

\subsubsection{Question 7}\label{question-7}

Calculate the average decrease in blood pressure for those where the
blood pressure decreased (positive change).

\begin{Shaded}
\begin{Highlighting}[]
\NormalTok{mean\_patients\_decreased\_bp }\OtherTok{\textless{}{-}} \FunctionTok{mean}\NormalTok{(patients\_decreased\_bp)}
\FunctionTok{paste}\NormalTok{(mean\_patients\_decreased\_bp)}
\end{Highlighting}
\end{Shaded}

\begin{verbatim}
[1] "20.6470588235294"
\end{verbatim}

\subsection{Task 2: Basic Data Frame
Practice}\label{task-2-basic-data-frame-practice}

\subsubsection{Question 1}\label{question-1-1}

Create a data frame object with four columns corresponding to your data
above: patient, pre\_bp, post\_bp, and diff\_bp.

\begin{Shaded}
\begin{Highlighting}[]
\NormalTok{bp\_df }\OtherTok{\textless{}{-}} \FunctionTok{data.frame}\NormalTok{(}\StringTok{"Patient"} \OtherTok{=}\NormalTok{ names, }
                    \StringTok{"pre\_bp"} \OtherTok{=}\NormalTok{ pre, }
                    \StringTok{"post\_bp"}\OtherTok{=}\NormalTok{post, }
                    \StringTok{"diff\_bp"}\OtherTok{=}\NormalTok{diff\_bp, }
                    \AttributeTok{row.names =} \ConstantTok{NULL}\NormalTok{)}
\end{Highlighting}
\end{Shaded}

\subsubsection{Question 2}\label{question-2-1}

Return only rows where the diff\_bp column is negative.

\begin{Shaded}
\begin{Highlighting}[]
\FunctionTok{subset}\NormalTok{(bp\_df, diff\_bp }\SpecialCharTok{\textless{}} \DecValTok{0}\NormalTok{)}
\end{Highlighting}
\end{Shaded}

\begin{verbatim}
      Patient pre_bp post_bp diff_bp
9   Subject_9    114     119      -5
13 Subject_13    128     130      -2
17 Subject_17    120     124      -4
\end{verbatim}

\subsubsection{Question 3}\label{question-3-1}

Add a new column to the data frame corresponding to TRUE if the post\_bp
is less than 120.

\begin{Shaded}
\begin{Highlighting}[]
\NormalTok{bp\_df}\SpecialCharTok{$}\NormalTok{post\_bp\_normal }\OtherTok{\textless{}{-}} \FunctionTok{ifelse}\NormalTok{(bp\_df}\SpecialCharTok{$}\NormalTok{post\_bp }\SpecialCharTok{\textless{}} \DecValTok{120}\NormalTok{, }\ConstantTok{TRUE}\NormalTok{, }\ConstantTok{FALSE}\NormalTok{)}
\end{Highlighting}
\end{Shaded}

Let's quickly check our logic worked.

\begin{Shaded}
\begin{Highlighting}[]
\FunctionTok{tail}\NormalTok{(bp\_df)}
\end{Highlighting}
\end{Shaded}

\begin{verbatim}
      Patient pre_bp post_bp diff_bp post_bp_normal
15 Subject_15    137     106      31           TRUE
16 Subject_16    131     106      25           TRUE
17 Subject_17    120     124      -4          FALSE
18 Subject_18    128     102      26           TRUE
19 Subject_19    139     117      22           TRUE
20 Subject_20    135     113      22           TRUE
\end{verbatim}

\subsubsection{Question 4}\label{question-4-1}

Finally, print the data frame out nicely in your final document by
modifying the code below appropriately.

\begin{Shaded}
\begin{Highlighting}[]
\NormalTok{knitr}\SpecialCharTok{::}\FunctionTok{kable}\NormalTok{(bp\_df)}
\end{Highlighting}
\end{Shaded}

\begin{longtable}[]{@{}lrrrl@{}}
\toprule\noalign{}
Patient & pre\_bp & post\_bp & diff\_bp & post\_bp\_normal \\
\midrule\noalign{}
\endhead
\bottomrule\noalign{}
\endlastfoot
Subject\_1 & 130 & 114 & 16 & TRUE \\
Subject\_2 & 128 & 98 & 30 & TRUE \\
Subject\_3 & 116 & 113 & 3 & TRUE \\
Subject\_4 & 124 & 99 & 25 & TRUE \\
Subject\_5 & 133 & 107 & 26 & TRUE \\
Subject\_6 & 134 & 116 & 18 & TRUE \\
Subject\_7 & 118 & 113 & 5 & TRUE \\
Subject\_8 & 126 & 111 & 15 & TRUE \\
Subject\_9 & 114 & 119 & -5 & TRUE \\
Subject\_10 & 127 & 117 & 10 & TRUE \\
Subject\_11 & 141 & 101 & 40 & TRUE \\
Subject\_12 & 138 & 119 & 19 & TRUE \\
Subject\_13 & 128 & 130 & -2 & FALSE \\
Subject\_14 & 140 & 122 & 18 & FALSE \\
Subject\_15 & 137 & 106 & 31 & TRUE \\
Subject\_16 & 131 & 106 & 25 & TRUE \\
Subject\_17 & 120 & 124 & -4 & FALSE \\
Subject\_18 & 128 & 102 & 26 & TRUE \\
Subject\_19 & 139 & 117 & 22 & TRUE \\
Subject\_20 & 135 & 113 & 22 & TRUE \\
\end{longtable}

\subsection{List Practice}\label{list-practice}

\subsubsection{Question 1}\label{question-1-2}

Create a new data frame with these data that is similar to the data
frame from task 2 (including the new column). That is, include a
patient, pre, post, diff, and normal (less than 120) column using the
data above. Name this new data frame bp\_df\_placebo.

\begin{Shaded}
\begin{Highlighting}[]
\CommentTok{\# I\textquotesingle{}m going to start the patient numbering after the previous names}
\NormalTok{names\_placebo }\OtherTok{\textless{}{-}} \FunctionTok{paste}\NormalTok{(}\StringTok{"Subject"}\NormalTok{, }\DecValTok{21}\SpecialCharTok{:}\DecValTok{30}\NormalTok{, }\AttributeTok{sep=}\StringTok{"\_"}\NormalTok{)}
\NormalTok{pre\_bp\_placebo }\OtherTok{\textless{}{-}} \FunctionTok{c}\NormalTok{(}\DecValTok{138}\NormalTok{, }\DecValTok{135}\NormalTok{, }\DecValTok{147}\NormalTok{, }\DecValTok{117}\NormalTok{, }\DecValTok{152}\NormalTok{, }\DecValTok{134}\NormalTok{, }\DecValTok{114}\NormalTok{, }\DecValTok{121}\NormalTok{, }\DecValTok{131}\NormalTok{, }\DecValTok{130}\NormalTok{)}
\NormalTok{post\_bp\_placebo }\OtherTok{\textless{}{-}} \FunctionTok{c}\NormalTok{(}\DecValTok{105}\NormalTok{, }\DecValTok{136}\NormalTok{, }\DecValTok{123}\NormalTok{, }\DecValTok{130}\NormalTok{, }\DecValTok{134}\NormalTok{, }\DecValTok{143}\NormalTok{, }\DecValTok{135}\NormalTok{, }\DecValTok{139}\NormalTok{, }\DecValTok{120}\NormalTok{, }\DecValTok{124}\NormalTok{)}

\NormalTok{bp\_df\_placebo }\OtherTok{\textless{}{-}} \FunctionTok{data.frame}\NormalTok{(}\StringTok{"Patient"} \OtherTok{=}\NormalTok{ names, }
                            \StringTok{"pre\_bp"} \OtherTok{=}\NormalTok{ pre\_bp\_placebo, }
                            \StringTok{"post\_bp"}\OtherTok{=}\NormalTok{post\_bp\_placebo, }
                            \AttributeTok{row.names =} \ConstantTok{NULL}\NormalTok{)}

\CommentTok{\# Let\textquotesingle{}s make the diff\_bp column a different way, this time from existing dataframe}
\NormalTok{bp\_df\_placebo}\SpecialCharTok{$}\NormalTok{diff\_bp }\OtherTok{\textless{}{-}}\NormalTok{ bp\_df\_placebo}\SpecialCharTok{$}\NormalTok{pre\_bp }\SpecialCharTok{{-}}\NormalTok{ bp\_df\_placebo}\SpecialCharTok{$}\NormalTok{post\_bp}

\CommentTok{\# Normal bp}
\NormalTok{bp\_df\_placebo}\SpecialCharTok{$}\NormalTok{post\_bp\_normal }\OtherTok{\textless{}{-}} \FunctionTok{ifelse}\NormalTok{(bp\_df\_placebo}\SpecialCharTok{$}\NormalTok{post\_bp }\SpecialCharTok{\textless{}} \DecValTok{120}\NormalTok{, }\ConstantTok{TRUE}\NormalTok{, }\ConstantTok{FALSE}\NormalTok{)}

\FunctionTok{head}\NormalTok{(bp\_df\_placebo)}
\end{Highlighting}
\end{Shaded}

\begin{verbatim}
    Patient pre_bp post_bp diff_bp post_bp_normal
1 Subject_1    138     105      33           TRUE
2 Subject_2    135     136      -1          FALSE
3 Subject_3    147     123      24          FALSE
4 Subject_4    117     130     -13          FALSE
5 Subject_5    152     134      18          FALSE
6 Subject_6    134     143      -9          FALSE
\end{verbatim}

\subsubsection{Question 2}\label{question-2-2}

Now create and store a list with two elements:

• 1st element named treatment and contains the first data frame you
created.

• 2nd element named placebo and contains the second data frame you
created.

\begin{Shaded}
\begin{Highlighting}[]
\NormalTok{bp\_list }\OtherTok{\textless{}{-}} \FunctionTok{list}\NormalTok{(}\StringTok{"treatment"} \OtherTok{=}\NormalTok{ bp\_df, }
                \StringTok{"placebo"} \OtherTok{=}\NormalTok{ bp\_df\_placebo)}
\end{Highlighting}
\end{Shaded}

\subsubsection{Question 3}\label{question-3-2}

Access the first list element using three different types of syntax.

\begin{Shaded}
\begin{Highlighting}[]
\CommentTok{\# By index}
\NormalTok{bp\_list[}\DecValTok{1}\NormalTok{]}
\end{Highlighting}
\end{Shaded}

\begin{verbatim}
$treatment
      Patient pre_bp post_bp diff_bp post_bp_normal
1   Subject_1    130     114      16           TRUE
2   Subject_2    128      98      30           TRUE
3   Subject_3    116     113       3           TRUE
4   Subject_4    124      99      25           TRUE
5   Subject_5    133     107      26           TRUE
6   Subject_6    134     116      18           TRUE
7   Subject_7    118     113       5           TRUE
8   Subject_8    126     111      15           TRUE
9   Subject_9    114     119      -5           TRUE
10 Subject_10    127     117      10           TRUE
11 Subject_11    141     101      40           TRUE
12 Subject_12    138     119      19           TRUE
13 Subject_13    128     130      -2          FALSE
14 Subject_14    140     122      18          FALSE
15 Subject_15    137     106      31           TRUE
16 Subject_16    131     106      25           TRUE
17 Subject_17    120     124      -4          FALSE
18 Subject_18    128     102      26           TRUE
19 Subject_19    139     117      22           TRUE
20 Subject_20    135     113      22           TRUE
\end{verbatim}

\begin{Shaded}
\begin{Highlighting}[]
\CommentTok{\# By index in the other direction}
\NormalTok{bp\_list[}\SpecialCharTok{{-}}\DecValTok{2}\NormalTok{]}
\end{Highlighting}
\end{Shaded}

\begin{verbatim}
$treatment
      Patient pre_bp post_bp diff_bp post_bp_normal
1   Subject_1    130     114      16           TRUE
2   Subject_2    128      98      30           TRUE
3   Subject_3    116     113       3           TRUE
4   Subject_4    124      99      25           TRUE
5   Subject_5    133     107      26           TRUE
6   Subject_6    134     116      18           TRUE
7   Subject_7    118     113       5           TRUE
8   Subject_8    126     111      15           TRUE
9   Subject_9    114     119      -5           TRUE
10 Subject_10    127     117      10           TRUE
11 Subject_11    141     101      40           TRUE
12 Subject_12    138     119      19           TRUE
13 Subject_13    128     130      -2          FALSE
14 Subject_14    140     122      18          FALSE
15 Subject_15    137     106      31           TRUE
16 Subject_16    131     106      25           TRUE
17 Subject_17    120     124      -4          FALSE
18 Subject_18    128     102      26           TRUE
19 Subject_19    139     117      22           TRUE
20 Subject_20    135     113      22           TRUE
\end{verbatim}

\begin{Shaded}
\begin{Highlighting}[]
\CommentTok{\# By name}
\NormalTok{bp\_list}\SpecialCharTok{$}\NormalTok{treatment}
\end{Highlighting}
\end{Shaded}

\begin{verbatim}
      Patient pre_bp post_bp diff_bp post_bp_normal
1   Subject_1    130     114      16           TRUE
2   Subject_2    128      98      30           TRUE
3   Subject_3    116     113       3           TRUE
4   Subject_4    124      99      25           TRUE
5   Subject_5    133     107      26           TRUE
6   Subject_6    134     116      18           TRUE
7   Subject_7    118     113       5           TRUE
8   Subject_8    126     111      15           TRUE
9   Subject_9    114     119      -5           TRUE
10 Subject_10    127     117      10           TRUE
11 Subject_11    141     101      40           TRUE
12 Subject_12    138     119      19           TRUE
13 Subject_13    128     130      -2          FALSE
14 Subject_14    140     122      18          FALSE
15 Subject_15    137     106      31           TRUE
16 Subject_16    131     106      25           TRUE
17 Subject_17    120     124      -4          FALSE
18 Subject_18    128     102      26           TRUE
19 Subject_19    139     117      22           TRUE
20 Subject_20    135     113      22           TRUE
\end{verbatim}

\subsubsection{Question 4}\label{question-4-2}

In one line, access the placebo data frame, pre\_bp column.

\begin{Shaded}
\begin{Highlighting}[]
\NormalTok{bp\_list}\SpecialCharTok{$}\NormalTok{placebo}\SpecialCharTok{$}\NormalTok{pre\_bp}
\end{Highlighting}
\end{Shaded}

\begin{verbatim}
 [1] 138 135 147 117 152 134 114 121 131 130 138 135 147 117 152 134 114 121 131
[20] 130
\end{verbatim}




\end{document}
